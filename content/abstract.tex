\cleardoublepage
\thispagestyle{plain}

\makeatletter
\begin{center}
	\large\textbf{\@title}\\
	\normalsize\@author
\end{center}
\makeatother

\paragraph{Abstract}

To shed some light on the construction of physically viable gravity theories, I employ the gravitational closure framework to investigate a cosmology that is not a priori based on a metric but assumes an area-metric spacetime geometry that arises from the most general linear theory of electrodynamics. To this end, I develop a symmetry reduction procedure to find, for the first time, an exact non-metric solution of the gravitational closure framework under cosmological symmetries. I investigate cosmological sources of gravity and construct an area-metric description of ideal fluids. I conclude by illustrating that, even though we cannot rely on metric concepts such as a metric line-element or a Levi-Civita connection to begin with, we can recover all required notions, such as light rays and observers, that allow us to conduct cosmological observations.

\begin{otherlanguage}{ngerman}

\paragraph{Zusammenfassung}

Mit dem Ziel zu unserem Verständnis beizutragen, wie physikalische Gravitationstheorien zu konstruieren sind, verwende ich das \emph{gravitational closure framework} um eine kosmologische Theorie zu untersuchen, die nicht auf einer metrischen, sondern flächenmetrischen Raumzeitgeometrie basiert. Eine solche liegt der allgemeinsten linearen Theorie der Elektrodynamik zugrunde. Ich entwickle dazu ein Symmetriereduktionsverfahren, um eine erste exakte, nicht-metrische Lösung für das \emph{gravitational closure framework} unter der Annahme kosmologischer Symmetrien herleiten zu können. Im Rahmen dieser Prozedur untersuche ich kosmologische Gravitationsquellen und konstruiere ideale Flüssigkeiten in flächenmetrischer Raumzeitgeometrie. Schließlich erläutere ich, wie wir, ohne mit metrischen Konzepten wie des Linienelements oder einer Levi-Civita Verbindung beginnen zu müssen, alle notwendigen Begriffe wie Lichtstrahlen und Beobachter wiederherstellen können, die wir benötigen, um kosmologische Messungen durchzuführen.

\end{otherlanguage}
